\documentclass[graybox,envcountchap,sectrefs]{svmono}

% choose options for [] as required from the list
% in the Reference Guide

\usepackage{mathptmx}
\usepackage{helvet}
\usepackage{courier}
%
\usepackage{type1cm}         

\usepackage{makeidx}         % allows index generation
\usepackage{graphicx}        % standard LaTeX graphics tool
                             % when including figure files
\usepackage{multicol}        % used for the two-column index
\usepackage[bottom]{footmisc}% places footnotes at page bottom
\usepackage{listings}
\usepackage{color}
 
\definecolor{codegreen}{rgb}{0,0.6,0}
\definecolor{codegray}{rgb}{0.5,0.5,0.5}
\definecolor{codepurple}{rgb}{0.58,0,0.82}
\definecolor{backcolour}{rgb}{0.95,0.95,0.92}
\lstdefinestyle{mystyle}{
    backgroundcolor=\color{backcolour},   
    commentstyle=\color{codegreen},
    keywordstyle=\color{magenta},
    numberstyle=\tiny\color{codegray},
    stringstyle=\color{codepurple},
    basicstyle=\footnotesize,
    breakatwhitespace=false,         
    breaklines=true,                 
    captionpos=b,                    
    keepspaces=true,                 
    numbers=left,                    
    numbersep=5pt,                  
    showspaces=false,                
    showstringspaces=false,
    showtabs=false,                  
    tabsize=2
}
\usepackage{url}
\lstset{style=mystyle}
\usepackage{sectsty} % Allows customizing section commands
\usepackage{longtable}
\usepackage{graphicx}
\usepackage{footmisc}
%\usepackage{cite}
\bibliographystyle{unsrt}
\graphicspath{{images/}{../images/}}
\usepackage{subfiles}

% see the list of further useful packages
% in the Reference Guide

\makeindex             % used for the subject index
                       % please use the style svind.ist with
                       % your makeindex program

%----------------------------------------------------------------------------------------
%	TITLE SECTION
%----------------------------------------------------------------------------------------


%%%%%%%%%%%%%%%%%%%%%%%%%%%%%%%%%%%%%%%%%%%%%%%%%%%%%%%%%%%%%%%%%%%%%

\begin{document}

\author{Simon Durbridge} % Your name

\title{Efficient Acoustic Modelling of Large Spaces using Time Domain Methods} % The assignment title

\subtitle{Analysis of Time Domain Numerical Methods for Acoustic Modelling of Large Spaces}
%\institute{Dept. Engineering, Mathematics \&\ Computing\\ University of Derby\\
%\email{s.durbridge1@unimail.derby.ac.uk} % optional, same for URL of
\maketitle

\frontmatter%%%%%%%%%%%%%%%%%%%%%%%%%%%%%%%%%%%%%%%%%%%%%%%%%%%%%%


%%%%%%%%%%%%%%%%%%%%%%% dedic.tex %%%%%%%%%%%%%%%%%%%%%%%%%%%%%%%%%
%
% sample dedication
%
% Use this file as a template for your own input.
%
%%%%%%%%%%%%%%%%%%%%%%%% Springer %%%%%%%%%%%%%%%%%%%%%%%%%%

\begin{dedication}
for Bethany
\end{dedication}





%%%%%%%%%%%%%%%%%%%%%%%foreword.tex%%%%%%%%%%%%%%%%%%%%%%%%%%%%%%%%%
% sample foreword
%
% Use this file as a template for your own input.
%
%%%%%%%%%%%%%%%%%%%%%%%% Springer %%%%%%%%%%%%%%%%%%%%%%%%%%

\foreword

%% Please have the foreword written here
Use the template \textit{foreword.tex} together with the Springer document class SVMono (monograph-type books) or SVMult (edited books) to style your foreword\index{foreword} in the Springer layout. 

The foreword covers introductory remarks preceding the text of a book that are written by a \textit{person other than the author or editor} of the book. If applicable, the foreword precedes the preface which is written by the author or editor of the book.


\vspace{\baselineskip}
\begin{flushright}\noindent
Place, month year\hfill {\it Firstname  Surname}\\
\end{flushright}



%%%%%%%%%%%%%%%%%%%%%%%preface.tex%%%%%%%%%%%%%%%%%%%%%%%%%%%%%%%%%%%%%%%%%
% sample preface
%
% Use this file as a template for your own input.
%
%%%%%%%%%%%%%%%%%%%%%%%% Springer %%%%%%%%%%%%%%%%%%%%%%%%%%

\preface

%% Please write your preface here
Use the template \emph{preface.tex} together with the Springer document class SVMono (monograph-type books) or SVMult (edited books) to style your preface in the Springer layout.

A preface\index{preface} is a book's preliminary statement, usually written by the \textit{author or editor} of a work, which states its origin, scope, purpose, plan, and intended audience, and which sometimes includes afterthoughts and acknowledgments of assistance. 

When written by a person other than the author, it is called a foreword. The preface or foreword is distinct from the introduction, which deals with the subject of the work.

Customarily \textit{acknowledgments} are included as last part of the preface.
 

\vspace{\baselineskip}
\begin{flushright}\noindent
Place(s),\hfill {\it Firstname  Surname}\\
month year\hfill {\it Firstname  Surname}\\
\end{flushright}



%%%%%%%%%%%%%%%%%%%%%%acknow.tex%%%%%%%%%%%%%%%%%%%%%%%%%%%%%%%%%%%%%%%%%
% sample acknowledgement chapter
%
% Use this file as a template for your own input.
%
%%%%%%%%%%%%%%%%%%%%%%%% Springer %%%%%%%%%%%%%%%%%%%%%%%%%%

\extrachap{Acknowledgements}
I would like to dedicate this work to anyone of remote importance.



\tableofcontents
%%%%%%%%%%%%%%%%%%%%%%acronym.tex%%%%%%%%%%%%%%%%%%%%%%%%%%%%%%%%%%%%%%%%%
% sample list of acronyms
%
% Use this file as a template for your own input.
%
%%%%%%%%%%%%%%%%%%%%%%%% Springer %%%%%%%%%%%%%%%%%%%%%%%%%%

\extrachap{Acronyms}

Use the template \emph{acronym.tex} together with the Springer document class SVMono (monograph-type books) or SVMult (edited books) to style your list(s) of abbreviations or symbols in the Springer layout.

Lists of abbreviations\index{acronyms, list of}, symbols\index{symbols, list of} and the like are easily formatted with the help of the Springer-enhanced \verb|description| environment.

\begin{description}[CABR]
\item[ABC]{Spelled-out abbreviation and definition}
\item[BABI]{Spelled-out abbreviation and definition}
\item[CABR]{Spelled-out abbreviation and definition}
\end{description}

\mainmatter%%%%%%%%%%%%%%%%%%%%%%%%%%%%%%%%%%%%%%%%%%%%%%%%%%%%%%%
\part{Background \& Theory}
\chapter{Introduction}

The intro Text

\section{Context}

\section{Problem Definition}
Real time acoustic modelling could be of significant benefit to many applications; Engineers could make design changes and see results 'on the fly', and entertainment users could have more realistic experiences. These benefits should be possible for an arbitrary number of sources and receivers, in proportionally large environments with high quality results. Is it possible to further reduce computation time for simulations of large acoustic problems, to provide results in real time for the full human audio frequency range? There are two 'branches' of computation solution that should be considered: the direct solution i.e. direct outputs or audio samples from the simulation, and indirect solutions i.e. a system impulse response that may be convolved with mixed source signals in order to create an auralization of the system.//


\begin{figure}
\centering
%\includegraphics[width=0.6\textwidth]{explicit2dfdtd.jpg}
\centering
\caption{A visualisation of a 2D explicit FDTD simulation ~\cite{Durbridge2016a}}
\end{figure}

\end{document} %Chapter
%%%%%%%%%%%%%%%%%%%%% chapter.tex %%%%%%%%%%%%%%%%%%%%%%%%%%%%%%%%%
%
% sample chapter
%
% Use this file as a template for your own input.
%
%%%%%%%%%%%%%%%%%%%%%%%% Springer-Verlag %%%%%%%%%%%%%%%%%%%%%%%%%%
%\motto{Use the template \emph{chapter.tex} to style the various elements of your chapter content.}
\chapter{Loudspeaker Systems \& Large Room Acoustics}

\abstract{Acoustics is a branch of physics that aims to characterise Newton's law of motion applied to wave propagation, while obeying the physical conservation law and often focussing on propagation in an audible spectrum. This characterisation of sound propagation is intrinsically linked to many other branches of physics, as well as psychoacoustics and perception. Many aspects of acoustic modelling may be of interest when considering the design and application of loudspeaker systems. Both small and large scale simulations may allow a user to make informed decisions about the design and placement of a loudspeaker system, so that the performance of the system may be validated and optimised before application. In this chapter we will evaluate the lossless acoustic wave equation for gasses, and consider the application of the wave equation in bounded space. We will then consider some specific use cases for applying such an equation for modelling loudspeaker system performance.}

\section{The Acoustic Wave Equation}
\label{sec:1}
In the Mcgraw-Hill Electronic and Electrical Engineering Series of books, the late L Beranek authored the Acoustics volume. This volume contains an elegant summary of the wave equation, that will be the subject of paraphrase in the following section.

\subsection{The Wave Equations}
Acoustic waves are classified as fluctuations of pressure in a given medium. In room acoustics and loudspeaker system engineering, these fluctuations are often cyclical in nature around an ambient pressure, as opposed to the jets described in aeroacoustic study. Similar to the behaviour of heat convection or fluid diffusion, these cyclical fluctuations propagate and spread through the medium of interest. As these fluctuations of pressure propagate energy is often lost, and eventually the medium will often come to a state of relative rest where the energy of the propagating waves have been almost entirely dissipated. It is possible to calculate an approximate solution the the propagation of pressure through a space, by solving a system of second order partial differential equations that can be collectively lumped into a 'Wave Equation'. Below, we will introduce the three building blocks of the wave equation in both one dimension, and three dimensions (based on vector notation). These building blocks are Newton's Second Law of Motion, the gas law, and the laws of conservation of mass.

To consider the wave equation, we should use the analogy of a small\footnote{rectilinear} volume of gas, within a larger homogeneous medium. The faces of the volume are frictionless, and only the pressure at any face impacts on the gas inside the volume.

\begin{center}
\begin{longtable}{|p{0.5\textwidth}|p{0.5\textwidth}|} 
  \hline
 One Dimension & Three Dimensions \\ [0.5ex] 
 \hline
Sound pressure $p$ propagates across the medium like a plane wave, from one side to the other in the $x$ direction at a rate equal to the change in space $\frac{\delta p}{\delta x}$  & Sound pressure $p$ propagates across the medium like a spherical wave, from one side to the other at a rate of \textbf{grad} $p = \textbf{i}\frac{\delta p}{\delta x} + \textbf{j}\frac{\delta p}{\delta y} + \textbf{k}\frac{\delta p}{\delta z}$ where i, j and k are unit vectors in the directions x, y and z.\\
\\
 Force acting on the volume in the positive $x$ direction can thus be described as $-(\frac{\delta p}{\delta x} \Delta x) \Delta y \Delta z$ & Force acting on the volume in the positive $x$ direction can thus be described as $-[i(\frac{\delta p}{\delta x} \Delta x) \Delta y \Delta z)+ j(\frac{\delta p}{\delta y} \Delta y) \Delta x \Delta z)+ k(\frac{\delta p}{\delta z} \Delta z) \Delta x \Delta y)]$\\
\\
 A positive gradient causes acceleration in the $-x$ direction & $\leftarrow$ \\
\\
 Force per unit volume is given by dividing both sides of the previous equation by the volume $V$, $\frac{\textit{f}}{V} = -\frac{\delta p}{\delta x}$ & Force per unit volume is given by dividing both sides of the previous equation by the volume $V$, $\frac{\textit{f}}{V} = - \textbf{grad} p$\\
\\
 Newton's second law of motion dictates that the rate of change of momentum in the volume must balance with force per unit volume, and we can assume the mass of gas in the volume is constant. & $ \leftarrow$ \\ [3.0ex]
The force mass balance can be described as $\frac{\textit{f}}{V} = -\frac{\delta p}{\delta x} = \frac{M}{M} \frac{\delta u}{\delta t} = \rho^{\prime}\frac{\delta u}{\delta t}$ & The force mass balance can be described as $\frac{\textit{f}}{V} = -\textbf{grad} p = \frac{M}{M} \frac{Dq}{Dt} = \rho^{\prime}\frac{Dq}{Dt}$\\ [1.5ex]
 $u$ is the velocity of gas in the volume, $\rho^{\prime}$ is the density of the gas, and $M = \rho^{\prime} V$ is the mass of gas in the volume.  & where $q$ is the vector velocity, $\rho^{\prime}$ is the density of gas in the volume, $M = \rho^{\prime} V$ is the total mass of gas in the volume. $\frac{D}{Dt}$ represents the total rate of change of velocity of a section of gas in the volume, and can be composed as $\frac{Dq}{Dt}=\frac{\delta q}{\delta t}+ q_x\frac{\delta q}{\delta x}+q_y\frac{\delta q}{\delta y}+q_z\frac{\delta q}{\delta z}$ where $q_x$, $q_y$ and $q_z$ are the components of the particle velocity \textbf{$q$} in each direction. As this is a linear wave equation approximation, these velocity components have no cross terms.\\
\\
If the change in density of gas in the volume is sufficiently small, the $\rho^{\prime}$ will be approximately equal to the average density $\rho_0$, thus simplifying the equations above to $-\frac{\delta p}{\delta x} = \rho_0 \frac{\delta u}{\delta t}$  & If the particle velocity vector is sufficiently small, the change of momentum of the gas is approximately the same as the momentum of the volume at any arbitrary point, and the density of gas within the volume $\rho^{\prime}$ will be approximately equal to the average density $\rho_0$. Thus the above can be written as $-grad p = \rho_0 \frac{\delta q}{\delta t}$ \\
\\
 This kind of approximation may be appropriate as long as the maximum pressure is appropriately low, so that the behaviour of the air is linear, often quoted to be at or under the threshold of pain for human hearing or 120dB SIL.  & $ \leftarrow$\\
\\
$ \rightarrow$ & Assuming that the gas of the volume is ideal, then the gas law $ PV = RT $ should hold true. Here, T is the temperature in degrees Kelvin, and R is a constant based on the mass of the gas. For this approximation we assume that the system is adiabatic, and that T and R are lumped into a gas constant which for air is $\gamma = 1.4$.  \\
\\
In differential form, the relationship between pressure and volume for an adiabatic expansion the volume is $\frac{dP}{P} = \frac{-\gamma dV}{V} $ i.e. changes in pressure scale with changes in volume by this $\gamma$ value. & $\leftarrow$ \\
\\
$\rightarrow$ & If perturbations in pressure and volume due to a sound wave, $p$ for pressure and $\tau$ for volume respectively, are sufficiently small compared to the rest values $P_0$ and $V_0$; the time based derivative of the above equation can be written as follows: $\frac{1}{P_0} \frac{\delta p}{\delta t} = \frac{-\gamma}{V_0} \frac{\delta \tau}{\delta t}$  \\
\\
As the wave equation being derived is concerned with the transport of pressure within a volume, a continuity expression must be applied. The conservation of mass states that the total mass of gas in the volume must remain constant. This conservation law brings a unique relationship between discrete velocities at the boundary of the volume: & $\leftarrow$\\
\\
If the volume is displaced by some rate $\epsilon_x$, air particles at either boundary of the volume must be displaced at an equal rate for the mass of the volume to remain constant. As such if the left side of the volume is displaced with a velocity, in a given time step the particles at the right hand boundary must also be displaced. This can be written as $\epsilon_x + \frac{\delta \epsilon_x}{\delta x} \Delta x$ The difference between this velocity and a subsequent change in volume $\tau$ multiplied by the volume gives $\tau = V_0\frac{\delta \epsilon_x}{\delta x}.$ & If the mass of gas within the box must remain constant, the vector displacement will directly change the volume by some rate, as the two must balance to satisfy the continuity equation. This can be written as $\tau = V_0$ $div$ $\epsilon $ \\
\\
Differentiating this with respect to time gives: $\frac{\delta \tau}{\delta t} = V_0\frac{\delta u}{\delta x}$ where u is the instantaneous particle velocity & Differentiating this with respect to time gives: $\frac{\delta \tau}{\delta t} = V_0$ $div$ $q$ where q is the instantaneous particle velocity\\
\\
The one dimensional wave equation in rectangular coordinates can be created by combining the above statements about the equation of motion, the gas law and the continuity equation. The combination of the gas law and continuity equation gives $\frac{\delta p}{\delta t} = -\gamma P_0\frac{\delta u}{\delta x}$ & The three dimensional wave equation in rectangular coordinates can be created by combining the above statements about the equation of motion, the gas law and the continuity equation. The combination of the gas law and continuity equation gives $\frac{\delta p}{\delta t} = -\gamma P_0 div \textbf{q}$\\
\\
When differentiated with respect to time, this gives: $\frac{\delta^2 p}{\delta t^2} = -\gamma P_0\frac{\delta^2 u}{\delta t \delta x}$ & When differentiated with respect to time this gives: $\frac{\delta^2 p}{\delta t^2} = -\gamma P_0 div \frac{\delta q}{\delta t}$\\
\\
Differentiating the momentum equation derived above with respect to time gives $-\frac{\delta^2 p}{\delta t^2}=\rho_0\frac{\delta^2 u}{\delta x \delta t}$ & The divergence of the momentum equation derived above gives:\ $-div = \rho_0 div \frac{\delta q}{\delta t}$\ Replacing the divergence $(grad p)$ term with the Lapacian operator$\nabla^2 p$ produces\ $-\nabla^2 p = \rho_0 div \frac{\delta^2 p}{\delta t}$\\
\\
Combining the above equations gives:\ $\frac{\delta^2 p}{\delta x^2}=\frac{\rho_0}{\gamma P_0}\frac{\delta^2 p}{\delta t^2} $ & Combining the above equations gives:\ $\nabla^2 p = \frac{\rho_0}{\gamma P_0}\frac{\delta^2 p}{\delta t^2}$ \\
\\
If we define c as the speed of propagation in the medium of interest, then $c^2\approx \frac{\gamma P_0}{\rho_0}$ due to the fact that the speed of sound $c \approx (1.4\frac{10^5}{1.18})^\frac{1}{2}$ where the ambient air pressure at sea level is $10^5Pa$, 1.4 is the adiabatic constant $\gamma$ (ratio of specific heats) for air, and $\rho_0$ is the density of air is approximately $1.8kg/m^3 $   & $\leftarrow$ \\
\\
Finally we find that the 1 dimensional wave equation is:\ $\frac{\delta^2 p}{\delta x^2}=\frac{1}{c^2}\frac{\delta^2 p}{\delta t^2}$ & Finally we find that the 3 dimensional wave equation is:\ $\nabla^2 p = \frac{1}{c^2}\frac{\delta^2 p}{\delta t^2}$\ An explicit 3 dimensional expression of the pressure component of this equation is:\ $\nabla^2 p = \frac{\delta^2p}{\delta x^2}+\frac{\delta^2p}{\delta y^2}+\frac{\delta^2p}{\delta z^2}$ \\
\\
This equation can also be expressed in terms of the instantaneous velocity in the volume as:\ $\frac{\delta^2 u}{\delta x^2}=\frac{1}{c^2}\frac{\delta^2 u}{\delta t^2}$ & This equation can be expressed velocity vector $\nabla^2 q = \frac{1}{c^2}\frac{\delta^2 q}{\delta t^2}$ where $\nabla^2 q$ represents the gradient of pressure (velocity) in the volume.\\
\hline
\end{longtable}
\end{center}


\section{Loudspeaker Systems}
\label{sec:2}
% Always give a unique label
% and use \ref{<label>} for cross-references
% and \cite{<label>} for bibliographic references
% use \sectionmark{}
% to alter or adjust the section heading in the running head
Instead of simply listing headings of different levels we recommend to let every heading be followed by at least a short passage of text. Furtheron please use the \LaTeX\ automatism for all your cross-references and citations.

Please note that the first line of text that follows a heading is not indented, whereas the first lines of all subsequent paragraphs are.

Use the standard \verb|equation| environment to typeset your equations, e.g.
%
\begin{equation}
a \times b = c\;,
\end{equation}
%
however, for multiline equations we recommend to use the \verb|eqnarray|
environment\footnote{In physics texts please activate the class option \texttt{vecphys} to depict your vectors in \textbf{\itshape boldface-italic} type - as is customary for a wide range of physical subjects.}.
\begin{eqnarray}
a \times b = c \nonumber\\
\vec{a} \cdot \vec{b}=\vec{c}
\label{eq:01}
\end{eqnarray}

\subsection{Subsection Heading}
\label{subsec:2}
Instead of simply listing headings of different levels we recommend to let every heading be followed by at least a short passage of text. Furtheron please use the \LaTeX\ automatism for all your cross-references\index{cross-references} and citations\index{citations} as has already been described in Sect.~\ref{sec:2}.

\begin{quotation}
Please do not use quotation marks when quoting texts! Simply use the \verb|quotation| environment -- it will automatically render Springer's preferred layout.
\end{quotation}


\subsubsection{Subsubsection Heading}
Instead of simply listing headings of different levels we recommend to let every heading be followed by at least a short passage of text. Furtheron please use the \LaTeX\ automatism for all your cross-references and citations as has already been described in Sect.~\ref{subsec:2}, see also Fig.~\ref{fig:1}\footnote{If you copy text passages, figures, or tables from other works, you must obtain \textit{permission} from the copyright holder (usually the original publisher). Please enclose the signed permission with the manucript. The sources\index{permission to print} must be acknowledged either in the captions, as footnotes or in a separate section of the book.}

Please note that the first line of text that follows a heading is not indented, whereas the first lines of all subsequent paragraphs are.

% For figures use
%
%\begin{figure}[b]
%\sidecaption
% Use the relevant command for your figure-insertion program
% to insert the figure file.
% For example, with the option graphics use
%\includegraphics[scale=.65]{figure}
%
% If not, use
%\picplace{5cm}{2cm} % Give the correct figure height and width in cm
%
%\caption{If the width of the figure is less than 7.8 cm use the \texttt{sidecapion} command to flush the caption on the left side of the page. If the figure is positioned at the top of the page, align the sidecaption with the top of the figure -- to achieve this you simply need to use the optional argument \texttt{[t]} with the \texttt{sidecaption} command}
%\label{fig:1}       % Give a unique label
%\end{figure}


\paragraph{Paragraph Heading} %
Instead of simply listing headings of different levels we recommend to let every heading be followed by at least a short passage of text. Furtheron please use the \LaTeX\ automatism for all your cross-references and citations as has already been described in Sect.~\ref{sec:2}.

Please note that the first line of text that follows a heading is not indented, whereas the first lines of all subsequent paragraphs are.

For typesetting numbered lists we recommend to use the \verb|enumerate| environment -- it will automatically render Springer's preferred layout.

\begin{enumerate}
\item{Livelihood and survival mobility are oftentimes coutcomes of uneven socioeconomic development.}
\begin{enumerate}
\item{Livelihood and survival mobility are oftentimes coutcomes of uneven socioeconomic development.}
\item{Livelihood and survival mobility are oftentimes coutcomes of uneven socioeconomic development.}
\end{enumerate}
\item{Livelihood and survival mobility are oftentimes coutcomes of uneven socioeconomic development.}
\end{enumerate}


\subparagraph{Subparagraph Heading} In order to avoid simply listing headings of different levels we recommend to let every heading be followed by at least a short passage of text. Use the \LaTeX\ automatism for all your cross-references and citations as has already been described in Sect.~\ref{sec:2}, see also Fig.~\ref{fig:2}.

Please note that the first line of text that follows a heading is not indented, whereas the first lines of all subsequent paragraphs are.

For unnumbered list we recommend to use the \verb|itemize| environment -- it will automatically render Springer's preferred layout.

\begin{itemize}
\item{Livelihood and survival mobility are oftentimes coutcomes of uneven socioeconomic development, cf. Table~\ref{tab:1}.}
\begin{itemize}
\item{Livelihood and survival mobility are oftentimes coutcomes of uneven socioeconomic development.}
\item{Livelihood and survival mobility are oftentimes coutcomes of uneven socioeconomic development.}
\end{itemize}
\item{Livelihood and survival mobility are oftentimes coutcomes of uneven socioeconomic development.}
\end{itemize}

%\begin{figure}[t]
%\sidecaption[t]
% Use the relevant command for your figure-insertion program
% to insert the figure file.
% For example, with the option graphics use
%\includegraphics[scale=.65]{figure}
%
% If not, use
%\picplace{5cm}{2cm} % Give the correct figure height and width in cm
%
%\caption{Please write your figure caption here}
%\label{fig:2}       % Give a unique label
%\end{figure}

\runinhead{Run-in Heading Boldface Version} Use the \LaTeX\ automatism for all your cross-references and citations as has already been described in Sect.~\ref{sec:2}.

\subruninhead{Run-in Heading Italic Version} Use the \LaTeX\ automatism for all your cross-refer\-ences and citations as has already been described in Sect.~\ref{sec:2}\index{paragraph}.
% Use the \index{} command to code your index words
%
% For tables use
%
\begin{table}
\caption{Please write your table caption here}
\label{tab:1}       % Give a unique label
%
% For LaTeX tables use
%
\begin{tabular}{p{2cm}p{2.4cm}p{2cm}p{4.9cm}}
\hline\noalign{\smallskip}
Classes & Subclass & Length & Action Mechanism  \\
\noalign{\smallskip}\svhline\noalign{\smallskip}
Translation & mRNA$^a$  & 22 (19--25) & Translation repression, mRNA cleavage\\
Translation & mRNA cleavage & 21 & mRNA cleavage\\
Translation & mRNA  & 21--22 & mRNA cleavage\\
Translation & mRNA  & 24--26 & Histone and DNA Modification\\
\noalign{\smallskip}\hline\noalign{\smallskip}
\end{tabular}
$^a$ Table foot note (with superscript)
\end{table}
%
\section{Section Heading}
\label{sec:3}
% Always give a unique label
% and use \ref{<label>} for cross-references
% and \cite{<label>} for bibliographic references
% use \sectionmark{}
% to alter or adjust the section heading in the running head
Instead of simply listing headings of different levels we recommend to let every heading be followed by at least a short passage of text. Furtheron please use the \LaTeX\ automatism for all your cross-references and citations as has already been described in Sect.~\ref{sec:2}.

Please note that the first line of text that follows a heading is not indented, whereas the first lines of all subsequent paragraphs are.

If you want to list definitions or the like we recommend to use the Springer-enhanced \verb|description| environment -- it will automatically render Springer's preferred layout.

\begin{description}[Type 1]
\item[Type 1]{That addresses central themes pertainng to migration, health, and disease. In Sect.~\ref{sec:1}, Wilson discusses the role of human migration in infectious disease distributions and patterns.}
\item[Type 2]{That addresses central themes pertainng to migration, health, and disease. In Sect.~\ref{subsec:2}, Wilson discusses the role of human migration in infectious disease distributions and patterns.}
\end{description}

\subsection{Subsection Heading} %
In order to avoid simply listing headings of different levels we recommend to let every heading be followed by at least a short passage of text. Use the \LaTeX\ automatism for all your cross-references and citations citations as has already been described in Sect.~\ref{sec:2}.

Please note that the first line of text that follows a heading is not indented, whereas the first lines of all subsequent paragraphs are.

\begin{svgraybox}
If you want to emphasize complete paragraphs of texts we recommend to use the newly defined Springer class option \verb|graybox| and the newly defined environment \verb|svgraybox|. This will produce a 15 percent screened box 'behind' your text.

If you want to emphasize complete paragraphs of texts we recommend to use the newly defined Springer class option and environment \verb|svgraybox|. This will produce a 15 percent screened box 'behind' your text.
\end{svgraybox}


\subsubsection{Subsubsection Heading}
Instead of simply listing headings of different levels we recommend to let every heading be followed by at least a short passage of text. Furtheron please use the \LaTeX\ automatism for all your cross-references and citations as has already been described in Sect.~\ref{sec:2}.

Please note that the first line of text that follows a heading is not indented, whereas the first lines of all subsequent paragraphs are.

\begin{theorem}
Theorem text goes here.
\end{theorem}
%
% or
%
\begin{definition}
Definition text goes here.
\end{definition}

\begin{proof}
%\smartqed
Proof text goes here.
\qed
\end{proof}

\paragraph{Paragraph Heading} %
Instead of simply listing headings of different levels we recommend to let every heading be followed by at least a short passage of text. Furtheron please use the \LaTeX\ automatism for all your cross-references and citations as has already been described in Sect.~\ref{sec:2}.

Note that the first line of text that follows a heading is not indented, whereas the first lines of all subsequent paragraphs are.
%
% For built-in environments use
%
\begin{theorem}
Theorem text goes here.
\end{theorem}
%
\begin{definition}
Definition text goes here.
\end{definition}
%
\begin{proof}
\smartqed
Proof text goes here.
\qed
\end{proof}
%
\begin{acknowledgement}
If you want to include acknowledgments of assistance and the like at the end of an individual chapter please use the \verb|acknowledgement| environment -- it will automatically render Springer's preferred layout.
\end{acknowledgement}
%
\section*{Appendix}
\addcontentsline{toc}{section}{Appendix}
%
When placed at the end of a chapter or contribution (as opposed to at the end of the book), the numbering of tables, figures, and equations in the appendix section continues on from that in the main text. Hence please \textit{do not} use the \verb|appendix| command when writing an appendix at the end of your chapter or contribution. If there is only one the appendix is designated ``Appendix'', or ``Appendix 1'', or ``Appendix 2'', etc. if there is more than one.

\begin{equation}
a \times b = c
\end{equation}
% Problems or Exercises should be sorted chapterwise
\section*{Problems}
\addcontentsline{toc}{section}{Problems}
%
% Use the following environment.
% Don't forget to label each problem;
% the label is needed for the solutions' environment
\begin{prob}
\label{prob1}
A given problem or Excercise is described here. The
problem is described here. The problem is described here.
\end{prob}

\begin{prob}
\label{prob2}
\textbf{Problem Heading}\\
(a) The first part of the problem is described here.\\
(b) The second part of the problem is described here.
\end{prob}

%%%%%%%%%%%%%%%%%%%%%%%%% referenc.tex %%%%%%%%%%%%%%%%%%%%%%%%%%%%%%
% sample references
% %
% Use this file as a template for your own input.
%
%%%%%%%%%%%%%%%%%%%%%%%% Springer-Verlag %%%%%%%%%%%%%%%%%%%%%%%%%%
%
% BibTeX users please use
% \bibliographystyle{}
% \bibliography{}
%
\bibliography{../library}
%\biblstarthook{In view of the parallel print and (chapter-wise) online publication of your book at \url{www.springerlink.com} it has been decided that -- as a genreral rule --  references should be sorted chapter-wise and placed at the end of the individual chapters. However, upon agreement with your contact at Springer you may list your references in a single seperate chapter at the end of your book. Deactivate the class option \texttt{sectrefs} and the \texttt{thebibliography} environment will be put out as a chapter of its own.\\\indent
%References may be \textit{cited} in the text either by number (preferred) or by author/year.\footnote{Make sure that all references from the list are cited in the text. Those not cited should be moved to a separate \textit{Further Reading} section or chapter.} The reference list should ideally be \textit{sorted} in alphabetical order -- even if reference numbers are used for the their citation in the text. If there are several works by the same author, the following order should be used: 
%\begin{enumerate}
%\item all works by the author alone, ordered chronologically by year of publication
%\item all works by the author with a coauthor, ordered alphabetically by coauthor
%\item all works by the author with several coauthors, ordered chronologically by year of publication.
%\end{enumerate}
%The \textit{styling} of references\footnote{Always use the standard abbreviation of a journal's name according to the ISSN \textit{List of Title Word Abbreviations}, see \url{http://www.issn.org/en/node/344}} depends on the subject of your book:
%\begin{itemize}
%\item The \textit{two} recommended styles for references in books on \textit{mathematical, physical, statistical and computer sciences} are depicted in ~\cite{science-contrib, science-online, science-mono, science-journal, science-DOI} and ~\cite{phys-online, phys-mono, phys-journal, phys-DOI, phys-contrib}.
%\item Examples of the most commonly used reference style in books on \textit{Psychology, Social Sciences} are~\cite{psysoc-mono, psysoc-online,psysoc-journal, psysoc-contrib, psysoc-DOI}.
%\item Examples for references in books on \textit{Humanities, Linguistics, Philosophy} are~\cite{humlinphil-journal, humlinphil-contrib, humlinphil-mono, humlinphil-online, humlinphil-DOI}.
%\item Examples of the basic Springer style used in publications on a wide range of subjects such as \textit{Computer Science, Economics, Engineering, Geosciences, Life Sciences, Medicine, Biomedicine} are ~\cite{basic-contrib, basic-online, basic-journal, basic-DOI, basic-mono}. 
%\end{itemize}
%}

%\begin{thebibliography}{99.}%
% and use \bibitem to create references.
%
% Use the following syntax and markup for your references if 
% the subject of your book is from the field 
% "Mathematics, Physics, Statistics, Computer Science"
%
% Contribution 
%\bibitem{science-contrib} Broy, M.: Software engineering --- from auxiliary to key technologies. In: Broy, M., Dener, E. (eds.) Software Pioneers, pp. 10-13. Springer, Heidelberg (2002)
%
% Online Document
%\bibitem{science-online} Dod, J.: Effective substances. In: The Dictionary of Substances and Their Effects. Royal Society of Chemistry (1999) Available via DIALOG. \\
%\url{http://www.rsc.org/dose/title of subordinate document. Cited 15 Jan 1999}
%
% Monograph
%\bibitem{science-mono} Geddes, K.O., Czapor, S.R., Labahn, G.: Algorithms for Computer Algebra. Kluwer, Boston (1992) 
%
% Journal article
%\bibitem{science-journal} Hamburger, C.: Quasimonotonicity, regularity and duality for nonlinear systems of partial differential equations. Ann. Mat. Pura. Appl. \textbf{169}, 321--354 (1995)
%
% Journal article by DOI
%\bibitem{science-DOI} Slifka, M.K., Whitton, J.L.: Clinical implications of dysregulated cytokine production. J. Mol. Med. (2000) doi: 10.1007/s001090000086 
%
%\bigskip

% Use the following (APS) syntax and markup for your references if 
% the subject of your book is from the field 
% "Mathematics, Physics, Statistics, Computer Science"
%
% Online Document
%\bibitem{phys-online} J. Dod, in \textit{The Dictionary of Substances and Their Effects}, Royal Society of Chemistry. (Available via DIALOG, 1999), 
%\url{http://www.rsc.org/dose/title of subordinate document. Cited 15 Jan 1999}
%
% Monograph
%\bibitem{phys-mono} H. Ibach, H. L\"uth, \textit{Solid-State Physics}, 2nd edn. (Springer, New York, 1996), pp. 45-56 
%
% Journal article
%\bibitem{phys-journal} S. Preuss, A. Demchuk Jr., M. Stuke, Appl. Phys. A \textbf{61}
%
% Journal article by DOI
%\bibitem{phys-DOI} M.K. Slifka, J.L. Whitton, J. Mol. Med., doi: 10.1007/s001090000086
%
% Contribution 
%\bibitem{phys-contrib} S.E. Smith, in \textit{Neuromuscular Junction}, ed. by E. Zaimis. Handbook of Experimental Pharmacology, vol 42 (Springer, Heidelberg, 1976), p. 593
%
%\bigskip
%
% Use the following syntax and markup for your references if 
% the subject of your book is from the field 
% "Psychology, Social Sciences"
%
%
% Monograph
%\bibitem{psysoc-mono} Calfee, R.~C., \& Valencia, R.~R. (1991). \textit{APA guide to preparing manuscripts for journal publication.} Washington, DC: American Psychological Association.
%
% Online Document
%\bibitem{psysoc-online} Dod, J. (1999). Effective substances. In: The dictionary of substances and their effects. Royal Society of Chemistry. Available via DIALOG. \\
%\url{http://www.rsc.org/dose/Effective substances.} Cited 15 Jan 1999.
%
% Journal article
%\bibitem{psysoc-journal} Harris, M., Karper, E., Stacks, G., Hoffman, D., DeNiro, R., Cruz, P., et al. (2001). Writing labs and the Hollywood connection. \textit{J Film} Writing, 44(3), 213--245.
%
% Contribution 
%\bibitem{psysoc-contrib} O'Neil, J.~M., \& Egan, J. (1992). Men's and women's gender role journeys: Metaphor for healing, transition, and transformation. In B.~R. Wainrig (Ed.), \textit{Gender issues across the life cycle} (pp. 107--123). New York: Springer.
%
% Journal article by DOI
%\bibitem{psysoc-DOI}Kreger, M., Brindis, C.D., Manuel, D.M., Sassoubre, L. (2007). Lessons learned in systems change initiatives: benchmarks and indicators. \textit{American Journal of Community Psychology}, doi: 10.1007/s10464-007-9108-14.
%
%
% Use the following syntax and markup for your references if 
% the subject of your book is from the field 
% "Humanities, Linguistics, Philosophy"
%
%\bigskip
%
% Journal article
%\bibitem{humlinphil-journal} Alber John, Daniel C. O'Connell, and Sabine Kowal. 2002. Personal perspective in TV interviews. \textit{Pragmatics} 12:257--271
%
% Contribution 
%\bibitem{humlinphil-contrib} Cameron, Deborah. 1997. Theoretical debates in feminist linguistics: Questions of sex and gender. In \textit{Gender and discourse}, ed. Ruth Wodak, 99--119. London: Sage Publications.
%
% Monograph
%\bibitem{humlinphil-mono} Cameron, Deborah. 1985. \textit{Feminism and linguistic theory.} New York: St. Martin's Press.
%
% Online Document
%\bibitem{humlinphil-online} Dod, Jake. 1999. Effective substances. In: The dictionary of substances and their effects. Royal Society of Chemistry. Available via DIALOG. \\
%http://www.rsc.org/dose/title of subordinate document. Cited 15 Jan 1999
%
% Journal article by DOI
%\bibitem{humlinphil-DOI} Suleiman, Camelia, Daniel C. O�Connell, and Sabine Kowal. 2002. `If you and I, if we, in this later day, lose that sacred fire...�': Perspective in political interviews. \textit{Journal of Psycholinguistic Research}. doi: 10.1023/A:1015592129296.
%
%
%
%\bigskip
%
%
% Use the following syntax and markup for your references if 
% the subject of your book is from the field 
% "Computer Science, Economics, Engineering, Geosciences, Life Sciences"
%
%
% Contribution 
%\bibitem{basic-contrib} Brown B, Aaron M (2001) The politics of nature. In: Smith J (ed) The rise of modern genomics, 3rd edn. Wiley, New York 
%
% Online Document
%\bibitem{basic-online} Dod J (1999) Effective Substances. In: The dictionary of substances and their effects. Royal Society of Chemistry. Available via DIALOG. \\
%\url{http://www.rsc.org/dose/title of subordinate document. Cited 15 Jan 1999}
%
% Journal article by DOI
%\bibitem{basic-DOI} Slifka MK, Whitton JL (2000) Clinical implications of dysregulated cytokine production. J Mol Med, doi: 10.1007/s001090000086
%
% Journal article
%\bibitem{basic-journal} Smith J, Jones M Jr, Houghton L et al (1999) Future of health insurance. N Engl J Med 965:325--329
%
% Monograph
%\bibitem{basic-mono} South J, Blass B (2001) The future of modern genomics. Blackwell, London 
%
%\end{thebibliography}

 %Chapter
%\include{sections/Numerical Methods} % Chapter
%%%%%%%%%%%%%%%%%%%%% chapter.tex %%%%%%%%%%%%%%%%%%%%%%%%%%%%%%%%%
%
% sample chapter
%
% Use this file as a template for your own input.
%
%%%%%%%%%%%%%%%%%%%%%%%% Springer-Verlag %%%%%%%%%%%%%%%%%%%%%%%%%%
%\motto{Use the template \emph{chapter.tex} to style the various elements of your chapter content.}
\chapter{Finite Difference Time Domain Method}
\label{Introduction}
The Finite Difference Time Domain Method is a numerical method for solving partial differential equations. The power of this method lies in its simplicity and flexibility, and it can be used to solve partial differential equations of varying complexity. In this chapter we will discuss the application of the finite difference time domain method to the acoustic wave equation, including the application of empirical partially absorbing boundary conditions.

\section{Introduction to the Finite Difference Time Domain Method}
%\label{sec:1}
Finite methods for solving partial differential equations have been of significant and continued research since the early 1900's, with mathematicians such as Courant, Fiedrichs and Hrennikof undertaking seminal work in the early 1920s, that formed a base for much of the finite methods used today. The Finite Difference Time Domain Method (FDTD) is a method for solving time domain problems (often wave equations) with loclised handling of time and space derivatives, and was first introduced for solving Maxwell's equations to simulate electromagnetic wave propagation by Yee \cite{Yee1966}. Yee proposed a method for which Maxwell's equations in partial differential form were applied to matrices staggered in partial steps in time and space, these matrices representing the magnetic (H) and electric (E) fields. In this explicit formulation, partial derivatives were used to solve H and E contiguously in a 'leapfrog' style, executing two sets of computations to solve for one time step. Multiple time steps would be solved from current time $t = 0$, in steps of $dt$ to the end of simulation time $T$.Each field is solved at half steps in time from each-other, thus H for a current time step $t + \delta t$ is calculated using the H values one time step ago $t$, and the E values half a time step ago $t + \frac{\delta t}{2} $. These two fields are also solved using central finite differences in space, in a staggered grid format i.e. E at index $x$ at time $t + \delta t$ is calculated using E at index $x$ at time $t$, and the finite difference between the local discrete values of H at $x - \frac{\delta x}{2} $ and at $x + \frac{\delta x}{2} $ at time $t + \frac{\delta t}{2}$. As such, it is possible to apply a simple kernel across many discretised points of a domain (H and E) to simulate electromagnetic wave propagation.

\section{The Finite Difference Time Domain Method Applied To The Acoustic Wave Equation}
%\label{sec:2}
The FDTD method applied to solving the acoustic wave equation, follows an almost identical form to that of solving Maxwells Equations with FDTD~\cite{Scheirman2015}\footnote{In fact, the equations follow an almost identical form}. Bottledooren's~\cite{Botteldooren1993} seminal work applied the FDTD method to the acoustic wave equations for both Cartesian and quasi-Cartesian grid systems. As previously described in the room acoustics section, the linear acoustic wave equation is based on Newton's second law of motion, the gas law and the continuity equation, and follows the form for the changes in the pressure and velocity respectively within a volume:\\
\begin{center}
$\frac{\delta^2 p}{\delta t^2} = \frac{1}{c^2} \frac{\delta^2 p}{\delta t^2}$\\
$\frac{\delta^2 u}{\delta t^2} = \frac{1}{c^2} \frac{\delta^2 u}{\delta t^2}$\\
\end{center}.
As pressure (p) and velocity (u) have a reciprocal relationship in a similar way to H and E, it is possible to rearrange the acoustic wave equation to reflect this relationship for a FDTD computation.

When treating the 1 dimensional linear acoustic wave equation with the FDTD method, it is possible to treat the p and u terms separately in time using the opposing terms for reciprocal calculation. As such, the p and u terms are reformulated as follows:\\
\begin{center}
$\frac{\delta^2 p}{\delta t^2} = p - \frac{\delta t}{\rho_0 \delta x} \frac{\delta^2 u}{\delta t^{2}}$\\
$\frac{\delta^2 u}{\delta t^2} = u - \frac{\delta t}{\rho_0 \delta x} \frac{\delta^2 u}{\delta t^{2}}$\\
\end{center}
However, this formulation is incomplete as it does not consider spatial or temporal discretisation of the field of interest, when applying the FDTD method. As the FDTD method relies on solving local finite difference approximations across a domain of interest, it is important to define a space and time index referencing method. In many mathematical texts, time step indexing is often represented by an i value, and spatial indexing often uses a j,k,l or l,m,n convention. For the aim of simplicity and as we will not directly address other forms of input output system in this text, we will use t for the time step indexing, and x, y and z for spatial indexing in each dimension.
Following an implementation of the acoustic FDTD method by Hill~\cite{Hill2012}, we can generate the following p and u equations for FDTD applied to the acoustic wave equation:\\
\begin{center}
$u^{t + \frac{\delta t}{2}}_{x} = u^{t - \frac{\delta t}{2}}_{x} - \frac{\delta t}{\rho \delta x} [p^{t}_{x + \frac{\delta x}{2}} - p^{t}_{x - \frac{\delta x}{2}}]$\\
$p^{t + \frac{\delta t}{2}}_{x} = p^{t - \frac{\delta t}{2}}_{x} - \frac{c^2 \rho \delta t}{\delta x} [u^{t}_{x + \frac{\delta x}{2}} - u^{t}_{x - \frac{\delta x}{2}}]$\\
\end{center}

Below, is a function written in the Matlab \textregistered language, used to solve one time step of the wave equation using the FDTD method, in 3 dimensions:
\lstinputlisting[language=Matlab]{../Matlab/FDTD/FDTD3Dfun.m}

\subsection{Stability}

\subsection{Handling Empirical Semi-Absorbing Boundary Conditions}

\section{A 2D Simulation with empirical partially absorbing boundary conditions}
%\label{sec:3}
% Always give a unique label
% and use \ref{<label>} for cross-references
% and \cite{<label>} for bibliographic references
% use \sectionmark{}
% to alter or adjust the section heading in the running head
Instead of simply listing headings of different levels we recommend to let every heading be followed by at least a short passage of text. Furtheron please use the \LaTeX\ automatism for all your cross-references and citations.

 %Chapter
%\include{sections/Sparse Finite Difference Time Domain Method} %Chapter
\include{sections/Pseudo-spectral Time Domain Method} %Chapter
\include{sections/Acoustic Modelling Strategies} %Chapter
\include{sections/Profiling} %Chapter
\include{sections/optimisationandparallelism} %Chapter
\include{sections/Analysis} %Chapter
\include{sections/Conclusion} %Chapter
\include{sections/Further Work} %Chapter
%%%%%%%%%%%%%%%%%%%%%% appendix.tex %%%%%%%%%%%%%%%%%%%%%%%%%%%%%%%%%
%
% sample appendix
%
% Use this file as a template for your own input.
%
%%%%%%%%%%%%%%%%%%%%%%%% Springer-Verlag %%%%%%%%%%%%%%%%%%%%%%%%%%

\appendix

\chapter{Code Listing}

On request of the supervisor for this project, program code listing is not included in this document. Instead, the programs can be found on the accompanying DVD and on Github at the following address:\\
\begin{center}
\url{https://github.com/SEDur/IndiEngiSchola}
\end{center}
\bibliography{library}

\backmatter%%%%%%%%%%%%%%%%%%%%%%%%%%%%%%%%%%%%%%%%%%%%%%%%%%%%%%%
%%%%%%%%%%%%%%%%%%%%%%%acronym.tex%%%%%%%%%%%%%%%%%%%%%%%%%%%%%%%%%%%%%%%%%
% sample list of acronyms
%
% Use this file as a template for your own input.
%
%%%%%%%%%%%%%%%%%%%%%%%% Springer %%%%%%%%%%%%%%%%%%%%%%%%%%

\Extrachap{Glossary}


Use the template \emph{glossary.tex} together with the Springer document class SVMono (monograph-type books) or SVMult (edited books) to style your glossary\index{glossary} in the Springer layout.


\runinhead{glossary term} Write here the description of the glossary term. Write here the description of the glossary term. Write here the description of the glossary term.

\runinhead{glossary term} Write here the description of the glossary term. Write here the description of the glossary term. Write here the description of the glossary term.

\runinhead{glossary term} Write here the description of the glossary term. Write here the description of the glossary term. Write here the description of the glossary term.

\runinhead{glossary term} Write here the description of the glossary term. Write here the description of the glossary term. Write here the description of the glossary term.

\runinhead{glossary term} Write here the description of the glossary term. Write here the description of the glossary term. Write here the description of the glossary term.
%
\Extrachap{Solutions}

\section*{Problems of Chapter~\ref{intro}}

\begin{sol}{prob1}
The solution\index{problems}\index{solutions} is revealed here.
\end{sol}


\begin{sol}{prob2}
\textbf{Problem Heading}\\
(a) The solution of first part is revealed here.\\
(b) The solution of second part is revealed here.
\end{sol}


\printindex

%%%%%%%%%%%%%%%%%%%%%%%%%%%%%%%%%%%%%%%%%%%%%%%%%%%%%%%%%%%%%%%%%%%%%%

\end{document}

