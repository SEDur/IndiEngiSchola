%%%%%%%%%%%%%%%%%%%%% chapter.tex %%%%%%%%%%%%%%%%%%%%%%%%%%%%%%%%%
%
% sample chapter
%
% Use this file as a template for your own input.
%
%%%%%%%%%%%%%%%%%%%%%%%% Springer-Verlag %%%%%%%%%%%%%%%%%%%%%%%%%%
%\motto{Use the template \emph{chapter.tex} to style the various elements of your chapter content.}
\chapter{Finite Difference Time Domain Method}
\label{Introduction}
The Finite Difference Time Domain Method is a numerical method for solving partial differential equations. The power of this method lies in its simplicity and flexibility, and it can be used to solve partial differential equations of varying complexity. In this chapter we will discuss the application of the finite difference time domain method to the acoustic wave equation, including the application of empirical partially absorbing boundary conditions.

\section{A Background to the Finite Difference Time Domain Method}
%\label{sec:1}
The Finite Difference Time Domain Method (FDTD) is a method first introduced by Yee, for solving Maxwell's equations in computational electromagnetics~\cite{Yee1966}. Yee proposed a method for which  

\section{The Finite Difference Time Domain Method Applied To The Wave Equation}
%\label{sec:2}
sometext

\section{A 2D Simulation with empirical partially absorbing boundary conditions}
%\label{sec:3}
% Always give a unique label
% and use \ref{<label>} for cross-references
% and \cite{<label>} for bibliographic references
% use \sectionmark{}
% to alter or adjust the section heading in the running head
Instead of simply listing headings of different levels we recommend to let every heading be followed by at least a short passage of text. Furtheron please use the \LaTeX\ automatism for all your cross-references and citations.

