%%%%%%%%%%%%%%%%%%%%% chapter.tex %%%%%%%%%%%%%%%%%%%%%%%%%%%%%%%%%
%
% sample chapter
%
% Use this file as a template for your own input.
%
%%%%%%%%%%%%%%%%%%%%%%%% Springer-Verlag %%%%%%%%%%%%%%%%%%%%%%%%%%
%\motto{Use the template \emph{chapter.tex} to style the various elements of your chapter content.}
\chapter{Conclusion and Further Work}
\section{Conclusion}
We introduced FDTD, PSTD and SFDTD.

We couldn't get it all going in 3D, but I tested SFDTD in 2D in as big a domain

One of them is faster, but FDTD isn't bad at all!


\section{Further Work}
\begin{itemize}
\item Single Precision Simulation on GPGPU for speed
\item Double Precision Simulation on GPGPU for accuracy
\item Object Oriented Implementation and Decomposition of Domains
\item Study of Obstacle Implementation in PSTD
\item Reducing the Gibbs Effect Cause By Partially Absorbing Boundaries in PSTD
\item Source Directivity
\item Convolutional PML and Frequency Dependent Boundary Conditions
\item Lossy Wave Equation in PSTD
\item C++ and Python Implementation
\item 3D SFDTD
\item Arbitrary geometry shapes and different meshing schemes
\item Rungakutta and Adams-Bashforth predictor correcter solving
\end{itemize}