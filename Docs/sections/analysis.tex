%%%%%%%%%%%%%%%%%%%%% chapter.tex %%%%%%%%%%%%%%%%%%%%%%%%%%%%%%%%%
%
% sample chapter
%
% Use this file as a template for your own input.
%
%%%%%%%%%%%%%%%%%%%%%%%% Springer-Verlag %%%%%%%%%%%%%%%%%%%%%%%%%%
%\motto{Use the template \emph{chapter.tex} to style the various elements of your chapter content.}
\chapter{Execution Time and Analysis}
\label{Introduction}
The interest of this study is to analyse the execution times of the three algorithms described above, to determine which method executes in the fastest time and thus might be most appropriate for using to solve large problems. The execution times will be measured for each method for domains of the following sizes at 10kHz sample rate:\\
\begin{center}
\begin{tabular}{|c|c|c|c|} 
  \hline
 Dimension ($m^2$) & FDTD Cells & SFDTDCells & PSTD Cells \\
 \hline
 2 & 121104	& 121104 & 24025\\ 
 4 & 483025 & 483025 & 62500\\  
 8 & 1932100 & 1932100 & 192721\\ 
 16 & 7722841 & 7722841 & 667489\\ 
 32 & 30880249 & 30880249 & 2474329\\ 
 64 & 123498769 & 123498769 & 9517225\\ 
 128 & 493950625 & 493950625 & 37319881\\ 
 256 & 1975802500 & 1975802500 & 147816964\\
 512 & 7903210000 & 7903210000 & 588353536\\
 \hline
\end{tabular}\\
\end{center}
As the implementation of SFDTD has not been successfully implemented  in 3D, all simulations will be performed in 2D. Matlabs Tic/Toc methods are used to calculate execution times within the loop, for $100ms$ of calculation time. The time steps for each method are:\\


\section{Results}
The average execution time for each domain is given below: \\

As domains get bigger, sfdtd and pstd dominate.


