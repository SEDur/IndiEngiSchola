%%%%%%%%%%%%%%%%%%%%% chapter.tex %%%%%%%%%%%%%%%%%%%%%%%%%%%%%%%%%
%
% sample chapter
%
% Use this file as a template for your own input.
%
%%%%%%%%%%%%%%%%%%%%%%%% Springer-Verlag %%%%%%%%%%%%%%%%%%%%%%%%%%
%\motto{Use the template \emph{chapter.tex} to style the various elements of your chapter content.}
\chapter{Acoustic Principals}
\label{Introduction}
Acoustics is a branch of physics\footnote{though often considered to be interdisciplinary} that aims to characterise Newton's law of motion applied to mechanical wave propagation, while obeying the physical conservation law and often focussing on propagation in an audible spectrum. This characterisation of sound propagation is intrinsically linked to many other disciplines of science and engineering, as well as psychological and perceptual study. In this section we will review the acoustic wave equation, and discuss some properties of interest in acoustic modelling.

%Many aspects of acoustic modelling may be of interest when considering the design and application of loudspeaker systems. Both small and large scale simulations may allow a user to make informed decisions about the design and placement of a loudspeaker system, so that the performance of the system may be validated and optimised before application. In this chapter we will evaluate the lossless acoustic wave equation for gasses, and consider the application of the wave equation in bounded space. We will then consider some specific use cases for applying such an equation for modelling loudspeaker system performance.

\section{The Acoustic Wave Equation}

In the Mcgraw-Hill Electronic and Electrical Engineering Series of books, the late Leo Beranek authored the Acoustics volume~\cite{beranek1954acoustics}. This volume contains an elegant summary of the wave equation, that will be the subject of paraphrase in the following section.\\

Acoustic waves are classified as fluctuations of pressure in a given medium, manifesting as longitudinal waves of high and low pressure and density of air molecules. Often these fluctuations are cyclical in nature around an ambient pressure, though jets are often described in aeroacoustic study. Similar to the behaviour of heat convection or fluid diffusion, these cyclical fluctuations propagate and spread through the medium of interest, converging towards an entropic steady state. It is possible to calculate an approximate solution to the propagation of pressure through a space, by solving a system of second order partial differential equations that can be collected into a 'Wave Equation'. Below, we will introduce the three building blocks of the wave equation in both one dimension, and three dimensions (based on vector notation). These building blocks are Newton's Second Law of Motion, the gas law, and the laws of conservation of mass.\\

To consider the wave equation, we should use the analogy of a small\footnote{rectilinear} volume of gas, within a larger homogeneous medium. The faces of the volume are frictionless, and only the pressure at any face impacts on the gas inside the volume.\\

\begin{center}
\begin{longtable}{|p{0.5\textwidth}|p{0.5\textwidth}|} 
  \hline
 One Dimension Standard & Three Dimension Vector \\ [0.5ex] 
 \hline
Sound pressure $p$ propagates across the medium like a plane wave, from one side to the other in the $x$ direction at a rate equal to the change in space $\frac{\delta p}{\delta x}$  & Sound pressure $p$ propagates across the medium like a spherical wave, from one side to the other at a rate of \textbf{grad} $p = \textbf{i}\frac{\delta p}{\delta x} + \textbf{j}\frac{\delta p}{\delta y} + \textbf{k}\frac{\delta p}{\delta z}$ where i, j and k are unit vectors in the directions x, y and z.\\
\\
 Force acting on the volume in the positive $x$ direction can thus be described as $-(\frac{\delta p}{\delta x} \Delta x) \Delta y \Delta z$ & Force acting on the volume in the positive $x$ direction can thus be described as $-[i(\frac{\delta p}{\delta x} \Delta x) \Delta y \Delta z)+ j(\frac{\delta p}{\delta y} \Delta y) \Delta x \Delta z)+ k(\frac{\delta p}{\delta z} \Delta z) \Delta x \Delta y)]$\\
\\
 A positive gradient causes acceleration in the $-x$ direction & $\leftarrow$ \\
\\
 Force per unit volume is given by dividing both sides of the previous equation by the volume $V$, $\frac{\textit{f}}{V} = -\frac{\delta p}{\delta x}$ & Force per unit volume is given by dividing both sides of the previous equation by the volume $V$, $\frac{\textit{f}}{V} = - \textbf{grad} p$\\
\\
 Newton's second law of motion dictates that the rate of change of momentum in the volume must balance with force per unit volume, and we can assume the mass of gas in the volume is constant. & $ \leftarrow$ \\ [3.0ex]
The force mass balance can be described as $\frac{\textit{f}}{V} = -\frac{\delta p}{\delta x} = \frac{M}{M} \frac{\delta u}{\delta t} = \rho^{\prime}\frac{\delta u}{\delta t}$ & The force mass balance can be described as $\frac{\textit{f}}{V} = -\textbf{grad} p = \frac{M}{M} \frac{Dq}{Dt} = \rho^{\prime}\frac{Dq}{Dt}$\\ [1.5ex]
 $u$ is the velocity of gas in the volume, $\rho^{\prime}$ is the density of the gas, and $M = \rho^{\prime} V$ is the mass of gas in the volume.  & where $q$ is the vector velocity, $\rho^{\prime}$ is the density of gas in the volume, $M = \rho^{\prime} V$ is the total mass of gas in the volume. $\frac{D}{Dt}$ represents the total rate of change of velocity of a section of gas in the volume, and can be composed as $\frac{Dq}{Dt}=\frac{\delta q}{\delta t}+ q_x\frac{\delta q}{\delta x}+q_y\frac{\delta q}{\delta y}+q_z\frac{\delta q}{\delta z}$ where $q_x$, $q_y$ and $q_z$ are the components of the particle velocity \textbf{$q$} in each direction. As this is a linear wave equation approximation, these velocity components have no cross terms.\\
\\
If the change in density of gas in the volume is sufficiently small, the $\rho^{\prime}$ will be approximately equal to the average density $\rho_0$, thus simplifying the equations above to $-\frac{\delta p}{\delta x} = \rho_0 \frac{\delta u}{\delta t}$  & If the particle velocity vector is sufficiently small, the change of momentum of the gas is approximately the same as the momentum of the volume at any arbitrary point, and the density of gas within the volume $\rho^{\prime}$ will be approximately equal to the average density $\rho_0$. Thus the above can be written as $-grad p = \rho_0 \frac{\delta q}{\delta t}$ \\
\\
 This kind of approximation may be appropriate as long as the maximum pressure is appropriately low, so that the behaviour of the air is linear, often quoted to be at or under the threshold of pain for human hearing or 120dB SIL.  & $ \leftarrow$\\
\\
$ \rightarrow$ & Assuming that the gas of the volume is ideal, then the gas law $ PV = RT $ should hold true. Here, T is the temperature in degrees Kelvin, and R is a constant based on the mass of the gas. For this approximation we assume that the system is adiabatic, and that T and R are lumped into a gas constant which for air is $\gamma = 1.4$.  \\
\\
In differential form, the relationship between pressure and volume for an adiabatic expansion the volume is $\frac{dP}{P} = \frac{-\gamma dV}{V} $ i.e. changes in pressure scale with changes in volume by this $\gamma$ value. & $\leftarrow$ \\
\\
$\rightarrow$ & If perturbations in pressure and volume due to a sound wave, $p$ for pressure and $\tau$ for volume respectively, are sufficiently small compared to the rest values $P_0$ and $V_0$; the time based derivative of the above equation can be written as follows: $\frac{1}{P_0} \frac{\delta p}{\delta t} = \frac{-\gamma}{V_0} \frac{\delta \tau}{\delta t}$  \\
\\
As the wave equation being derived is concerned with the transport of pressure within a volume, a continuity expression must be applied. The conservation of mass states that the total mass of gas in the volume must remain constant. This conservation law brings a unique relationship between discrete velocities at the boundary of the volume: & $\leftarrow$\\
\\
If the volume is displaced by some rate $\epsilon_x$, air particles at either boundary of the volume must be displaced at an equal rate for the mass of the volume to remain constant. As such if the left side of the volume is displaced with a velocity, in a given time step the particles at the right hand boundary must also be displaced. This can be written as $\epsilon_x + \frac{\delta \epsilon_x}{\delta x} \Delta x$ The difference between this velocity and a subsequent change in volume $\tau$ multiplied by the volume gives $\tau = V_0\frac{\delta \epsilon_x}{\delta x}.$ & If the mass of gas within the box must remain constant, the vector displacement will directly change the volume by some rate, as the two must balance to satisfy the continuity equation. This can be written as $\tau = V_0$ $div$ $\epsilon $ \\
\\
Differentiating this with respect to time gives: $\frac{\delta \tau}{\delta t} = V_0\frac{\delta u}{\delta x}$ where u is the instantaneous particle velocity & Differentiating this with respect to time gives: $\frac{\delta \tau}{\delta t} = V_0$ $div$ $q$ where q is the instantaneous particle velocity\\
\\
The one dimensional wave equation in rectangular coordinates can be created by combining the above statements about the equation of motion, the gas law and the continuity equation. The combination of the gas law and continuity equation gives $\frac{\delta p}{\delta t} = -\gamma P_0\frac{\delta u}{\delta x}$ & The three dimensional wave equation in rectangular coordinates can be created by combining the above statements about the equation of motion, the gas law and the continuity equation. The combination of the gas law and continuity equation gives $\frac{\delta p}{\delta t} = -\gamma P_0 div \textbf{q}$\\
\\
When differentiated with respect to time, this gives: $\frac{\delta^2 p}{\delta t^2} = -\gamma P_0\frac{\delta^2 u}{\delta t \delta x}$ & When differentiated with respect to time this gives: $\frac{\delta^2 p}{\delta t^2} = -\gamma P_0 div \frac{\delta q}{\delta t}$\\
\\
Differentiating the momentum equation derived above with respect to time gives $-\frac{\delta^2 p}{\delta t^2}=\rho_0\frac{\delta^2 u}{\delta x \delta t}$ & The divergence of the momentum equation derived above gives:\ $-div = \rho_0 div \frac{\delta q}{\delta t}$\ Replacing the divergence $(grad p)$ term with the Lapacian operator$\nabla^2 p$ produces\ $-\nabla^2 p = \rho_0 div \frac{\delta^2 p}{\delta t}$\\
\\
Combining the above equations gives:\ $\frac{\delta^2 p}{\delta x^2}=\frac{\rho_0}{\gamma P_0}\frac{\delta^2 p}{\delta t^2} $ & Combining the above equations gives:\ $\nabla^2 p = \frac{\rho_0}{\gamma P_0}\frac{\delta^2 p}{\delta t^2}$ \\
\\
If we define c as the speed of propagation in the medium of interest, then $c^2\approx \frac{\gamma P_0}{\rho_0}$ due to the fact that the speed of sound $c \approx (1.4\frac{10^5}{1.18})^\frac{1}{2}$ where the ambient air pressure at sea level is $10^5Pa$, 1.4 is the adiabatic constant $\gamma$ (ratio of specific heats) for air, and $\rho_0$ is the density of air is approximately $1.8kg/m^3 $   & $\leftarrow$ \\
\\
Finally we find that the 1 dimensional wave equation is:\ $\frac{\delta^2 p}{\delta x^2}=\frac{1}{c^2}\frac{\delta^2 p}{\delta t^2}$ & Finally we find that the 3 dimensional wave equation is:\ $\nabla^2 p = \frac{1}{c^2}\frac{\delta^2 p}{\delta t^2}$\ An explicit 3 dimensional expression of the pressure component of this equation is:\ $\nabla^2 p = \frac{\delta^2p}{\delta x^2}+\frac{\delta^2p}{\delta y^2}+\frac{\delta^2p}{\delta z^2}$ \\
\\
This equation can also be expressed in terms of the instantaneous velocity in the volume as:\ $\frac{\delta^2 u}{\delta x^2}=\frac{1}{c^2}\frac{\delta^2 u}{\delta t^2}$ & This equation can be expressed velocity vector $\nabla^2 q = \frac{1}{c^2}\frac{\delta^2 q}{\delta t^2}$ where $\nabla^2 q$ represents the gradient of pressure (velocity) in the volume.\\
\hline
\end{longtable}
\end{center}

In the above table we have derived wave equations, with forms of velocity and pressure as the independent variables. We have also shown that pressure, velocity, displacement and density are related within the system of equations, by differentiating and integrating with respect to space and time. As these forms of the wave equation are intrinsically coupled, it is possible to leverage this coupling when generating a numerical solution to the wave equation. It is also important to note that a significant number of assumptions have been taken when deriving these equations, and any solution to these equations may only be accurate when simulating a loss free, frictionless, homogeneous, ideal gas medium, where all perturbations are sufficiently small and fast that it is possible to reduce the complexity of the system.\\

\section{Acoustic Properties of Interest in Basic Simulations}
Now that we have an understanding of the mathematics behind sound propagation from the wave equation, it is important to have an understanding of what acoustic phenomena can be observed through solving the wave equation. In the next section we shall discuss three components of acoustics behaviour, two of which are intrinsic to the acoustics of rooms and one is more general.\\

\subsection{Inverse Square Law \& Propagation time}
As previously noted, sound propagates as longitudinal waves through a medium such as air or water. These waves are often conceptualised as simple rays travelling through a space\footnote{It may be appropriate to always consider space to be 3 dimensional (3D), or a lower order approximation of a 3D space}, much like planar waves. However, the properties of a sound source such as the directivity and shape, can have a significant effect on the behaviour of sound wave propagation. An example of this is the difference in energy spread over distance for theoretically ideal point and line sources. Ideal point sources that propagate sound omni-directionally obey the inverse square law, and ideal line sources do not, as they propagate sound cylindrically. The inverse square law is sound propagations is defined as:\\
\begin{center}
$I = \frac{P}{4 \pi r^2}$\\
\end{center}
Where I is the intensity over the area of the sphere, P is the propagated energy at the source and r is the radius of the sphere i.e. the distance between the source and the point of inquiry.\\



However, as line sources propagate cylindrically, the equation above can be modified to account for this change:\\
\begin{center}
$I = \frac{P}{2 \pi r}$\\
\end{center}
Below is a graph showing the difference between intensity over distance for an ideal point and line source:\\

Although the wave equation considered and solved in this study is lossless i.e. we do not consider viscous or thermal losses in the basic linearised acoustic wave equation, we would expect to see a reduction in absolute pressure between a source and receiver. As sound travels at some finite distance over time $c$, we would also expect to see a uniform time between a wave being radiated from a source, and being recorded at some receiver location for all simulation methods\footnote{For an interesting review of the relationship between 1D, 2D and 3D sound propagation being derivative, please see the appendices}.\\ 

\subsection{Reverberation}
For this study we will consider spaces or domains of finite size. These domains have boundaries, and those boundaries will either absorb or reflect sound waves. In acoustic engineering, the proportion of sound energy absorbed or reflected by a material is often described as an absorption coefficient, and is a normalised value between 0 (totally reflecting) and 1 (totally absorbing). If a sound source propagates a signal of appropriate speed and amplitude, the sound wave will reach the boundaries and be partially absorbed or reflected in reciprocal directions, and these reflections will scatter and eventually decay beyond audibility. The reverberant sound field is the steady-state of diffusely scattered sound energy in a space due to the high order reflection from boundaries. The amplitude of the diffusely scattered reflections are such as to balance in amplitude with the rate of decay and absorption of the sound field~\cite{Everest2009}.\\ 

%\begin{figure}[H]
%\centering
%\includegraphics[width=0.4\textwidth]{reflection_diagran.jpg}
%\centering
%\caption{A conceptualisation of direct and reflected sound~\cite{Everest2009}}
%\end{figure}

Low order reflections often described as early reflections in relation to psychoacoustics, may occur above the steady state amplitude (echos) ~\cite{Everest2009} if the steady state amplitude decreases appropriately. Early and strong reflections are of significant interest in the auralization and perception of sound fields (auditory scene analysis), due to the cues humans receive from perception of them e.g. room size and source direction information. The decay rate of a reverberant sound field is often quantified by the time taken for a steady state sound field to reduce in level by $60{dB}$, once the sound source has finished propagating. This is defined as the $RT_{60}$ of the domain, and was first proposed by WC Sabine in 1900~\cite{Everest2009}. There have been a multitude of expansions on Sabines original formula, notably in this study by Eyring, who expanded the denominator of the reverberation time equation to calculate more realistically for average absorption values above $0.1$. The Eyring reverberation time equation is as follows:\\



The use of $RT_{60}$ as the preferred metric of decay time is valid, assuming that the acoustics system is linear and time-invariant.
A more comprehensive description of reverberation and overview of the associated parameters is given by Rossing~\cite{rossing2007springer}. 


\subsubsection{Acoustic Absorption}

\subsection{Room Modes}

\subsubsection{Schroeder Frequency}

\subsection{Perception of Early Reflections}

\subsection{Auditory Scene Analysis}


