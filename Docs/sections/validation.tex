%%%%%%%%%%%%%%%%%%%%% chapter.tex %%%%%%%%%%%%%%%%%%%%%%%%%%%%%%%%%
%
% sample chapter
%
% Use this file as a template for your own input.
%
%%%%%%%%%%%%%%%%%%%%%%%% Springer-Verlag %%%%%%%%%%%%%%%%%%%%%%%%%%
%\motto{Use the template \emph{chapter.tex} to style the various elements of your chapter content.}
\chapter{Validation}
\label{Introduction}
While it may be beneficial and interesting to review the behaviour of a wave propagating in a fictitious or simulated domain, model validation could be considered an important step towards creating a such a robust and useful tool. Below we shall discuss a scenario that is used for the validation of the simulation tools described in this study, and we shall review the performance of such tools in comparison to the hand calculated properties of the scenario and the results of an Image Source model.  

\section{A Model Environment}
The model environment used for validation in this study shall be a fully enclosed room of the following dimensions:\\
%\begin{center}
\begin{tabular}{|c|c|} 
  \hline
 Dimension & Length (m) \\
 \hline
 x & 5 \\ 
 y & 4 \\  
 z & 3 \\  
 \hline
\end{tabular}\\
%\end{center}
This gives a volume of $ v = 60m^3$, and a boundary surface area of $S = 94m^2$.\\
The boundaries shall have a uniform absorption coefficient of $\alpha = 0.45 $. As the boundaries are uniformly absorbing and the coefficient average is above 0.1, it may be appropriate to use the Eyring reverberation time equation which yields $RT_{60} = 0.1719s $.\\
The average number of reflections before the energy of a wave-front has decayed below the noise floor will be $N_{reflections} = 30.7$, and the mean free path between reflections will be $MFP = 2.55m$.\\
The Schroeder frequency of the room will be $f_{schroeder} = 107Hz $, and the axial, tangential and oblique modes below the Schroeder frequency are calculated as follows:\\

\section{Results}




