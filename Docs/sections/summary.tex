
%%%%%%%%%%%%%%%%%%%%%%% dedic.tex %%%%%%%%%%%%%%%%%%%%%%%%%%%%%%%%%
%
% sample dedication
%
% Use this file as a template for your own input.
%
%%%%%%%%%%%%%%%%%%%%%%%% Springer %%%%%%%%%%%%%%%%%%%%%%%%%%

\chapter*{Summary}
Time domain numerical for solving acoustic equations have been of continuing interest and development since early work by key figures such as Bottledooren~\cite{Botteldooren1995}. Some distinct disadvantages to these methods mean that they are often implemented only for low frequency modelling, and in hybrid schemes such as those presented by Mourik and Murphy ~\cite{Mourik2014a}. Ray based acoustic modelling methods may be no more accurate of efficient when modelling large domains, as the number of rays required for accurate simulation can become very large. In this study three time domain numerical methods are implemented, the finite difference time domain method, the sparse finite difference time domain method and the psudospectral time domain method. The execution times for these three methods are evaluated in domains of varying size, using commonly available computing resources. It is found that the pseudospectral time domain method is the fastest per time step of the three methods, when solving for very large domains i.e. millions cells in volume. This is potentially due to the nature of differentiation by multiplication in the frequency domain.

\chapter*{Acknowledgements}

I would like to thank my supervisor Dr Adam Hill for his friendly, diligent and understanding guidance throughout my academic career. He has guided and steadied my hand through all circumstances even through my often difficult character, and it is on his work that this work has evolved. His honest and poignant observations have helped my confidence immensely, and for his help I am exceedingly grateful.\\

I would like to thank my family and my partner Bethany, for all of their help, support and understanding throughout the course of this masters degree. Without Beths constant and accommodating support, I would be nothing more than a ruined mess and this work wouldn't have even been started.

I would like to thank all those whose work I have built from, and those who have provided guidance. Particularly Jamie Angus, Damian Murphy and Raymond Rumpf. I would also like to thank the team at Bowers and Wilkins for their help and support.

Finally I would one again like to thank and congratulate William and Yasmin Draffin. Without these two I would not have entered the world of academia, and I wouldn't be where I am today.